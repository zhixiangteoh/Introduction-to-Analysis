\documentclass[11pt,twoside, reqno]{amsart}

%%%%%%%%%%%%%%%%%%%%%%%%%%%%%%%packages%%%%%%%%%%%%%%%%%%%


%%%%%%%%%%%%%%%%%%%%%%%%%%%%%%%formatting%%%%%%%%%%%%%%%%%%

\setlength{\topmargin}{0in} 
\setlength{\oddsidemargin}{0in}   
\setlength{\evensidemargin}{0in}  
\setlength{\textheight}{8.5in}    
\setlength{\textwidth}{6.5in}  
\setlength{\headsep}{0.5in}   
\setlength{\headheight}{0in}
\parskip=4pt 

%%%%%%%%%%%%%%%%%%%%%%%%%%%%%%%formatting%%%%%%%%%%%%%%%%%%

\newtheorem{Thm}{Theorem}
\newtheorem{Def}[Thm]{Definition}
\newtheorem{Lm}[Thm]{Lemma}
\newtheorem{Prop}[Thm]{Proposition}
\newtheorem{Cor}[Thm]{Corollary}


\theoremstyle{remark}
\newtheorem{Rem}[Thm]{Remark}
\newtheorem{Exp}[Thm]{Example}
\newtheorem{Prob}{Problem}

%\numberwithin{equation}{section}



\def\R{\mathbb R}
\def\Q{\mathbb Q}
\def\N{\mathbb N}
\def\Z{\mathbb Z}


%%%%%%%%%%%%%%%%logical connectors%%%%%%%%%%%%%%%%%%%%%%%%%%%%%%%%%%%%%

\newcommand{\OR}{\vee}
\newcommand{\AND}{\wedge}
\renewcommand{\implies}{\Rightarrow}
\newcommand{\implied}{\Leftarrow}
\renewcommand{\iff}{\Leftrightarrow}

%%%%%%%%%%%%%%%%%%%%%%%%%%%%%%%%%%%%%%%%%%%%%%%%%%%%%

\begin{document}

\title{Math 0450: Homework 5}
\date{\today}
\author{Teoh Zhixiang}

\maketitle



\begin{Prob}
Show that the function $f:\N\times \N\to \N$, defined by $f(a,b)=(a+b)(a+b+1)/2+b$ is bijective. 

\end{Prob}

\begin{proof}
Write your solution here.

\end{proof}

\begin{Prob}(Ex. 1.5.1) Finish the proof for Theorem 1.5.7: If $A\subseteq B$ and $B$ is countable, then $A$ is either countable or finite.
\end{Prob}

\begin{proof}
Write your solution here.

\end{proof}


\begin{Prob}(Ex. 1.5.2) Use the following outline (as specified in the textbook) to supply proofs for the statements in Theorem 1.5.8.
\end{Prob}

\begin{proof}
Write your solution here.

\end{proof}


\begin{Prob}(Ex. 1.5.4) \begin{enumerate}
\item[(a)] Show $(a, b) \sim \R$ for any interval $(a, b)$.
\item[(b)] Show that an unbounded interval like $(a, \infty) = \{x~ :~ x > a\}$ has the same cardinality as $\R$ as well.
\item[(c)] Using open intervals makes it more convenient to produce the required 1--1, onto functions, but it is not really necessary. Show that $[0, 1)\sim (0, 1)$ by exhibiting a 1--1 onto function between the two sets.
\end{enumerate}
\end{Prob}

\begin{proof}
Write your solution here.

\end{proof}

\begin{Prob}(Ex. 1.5.6 (b)) Give an example of an uncountable collection of disjoint open intervals, or argue that no such collection exists.
\end{Prob}

\begin{proof}
Write your solution here.

\end{proof}


\end{document}
